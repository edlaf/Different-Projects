\documentclass{article}
\usepackage{graphicx} % Required for inserting images
\usepackage{graphicx} % Requis pour l'insertion d'images
\usepackage{geometry}
\usepackage{amssymb}
\usepackage{amsmath}
\geometry{hmargin=2.5cm,vmargin=1.3cm}
\title{Quasi-Orthogonal Matrix Generation}
\author{Edouard Laferté, Elias Chouik}
\date{March 2025}

\begin{document}

\maketitle

We focus on the generation of quasi-Orthogonal matrix, for computing purposes. For instance,
these matrices are used to do projection in high dimension. In particular, we use these matrices
in the sparce projection theorem, where the quasi-orthogonal matrix is construted as the limit of
a random matrix.
\\
\\
\textbf{Definition: }$\epsilon$ Quasi-orthogonal matrix.
\\
\\
A $\epsilon$ Quasi-orthogonal matrix is a matrix $A\in \mathcal M_{m,n}$ such that:
$$||A^TA-I_{n}||_{\infty}\leq \epsilon$$
This implies that $$\forall j \in [1,n], \ 1-\epsilon \leq ||A_j^TA_j||_{\infty}\leq 1+\epsilon$$
and $$\forall (i,j) \in [1,n]^2, \ i\neq j, ||A_i^TA_j||_{\infty}\leq \epsilon$$

\section{Random matrices}
We first focus on the generation of such matrices using random matrices.
\\
\\
\textbf{Proposition:}
Let $(X_1,...,X_n)\in \mathbb R^n_m$ random variables indentically distributed and of covariance matrice $I_n$, with a 
probability distribution that have a mean and a variance.
Let name $X^{(n)}$ the matrix:
$$X^{(n)} = (X_1|\dots|X_n)$$
We have using the \textit{Law of Large Number}
$$X^{(n)}^TX^{(n)}\underset{n \xrightarrow{} \infty}{\xrightarrow{}} I_n \ \text{a.s}$$
\\
\\
It is thus rather easy to create a quasi-orthogonal matrix. Sample each $X_i$ from a distribution with the past criterion
and its asymptotic covariance matrix will have the needed caracterists almost certainly.
\\
\\
\textbf{Proposition:}
The normal law is the best probability distribution possible considering all distribution having an infinit number of moment.
\\
\\
\textit{Proof:}
The proof is direct using Edgeworth serie formula with a probability that has a different cumulant than the normal law.
\\
\\
Thus, we can see that it is not possible to find a better probability to sample from if we stick to this strategy. However,
other theorems exists surrounding random matrices such as the quarter-circle law theorem for semi-definitve matrix or the
Marchenko-Pastur theorem.
\section{Non Random technics}
The set of matrix that are $0$-quasi-orthogonal matrix is called the Stiefeld manifold. We need to find some caracterists about this
manifold.
\end{document}